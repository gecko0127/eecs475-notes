\documentclass{scribe}

%====================================================================
%====================================================================
% IMPORTANT: PLEASE UPDATE THE LECTURE INFORMATION BELOW:
\setcounter{lecture}{24} % Lecture number here, from The Big Table on Canvas
\renewcommand{\lectureTitle}{Digital Signatures, Modeling Digital Signatures, RSA Signatures}
\renewcommand{\lecturer}{Mahdi Cheraghchi}
\renewcommand{\scribe}{Yi-Wen Tseng}
\renewcommand{\lectureDate}{April 5, 2023} % Date of the lecture
%====================================================================
%====================================================================

\begin{document}

\maketitle

%=============================================================================
%=============================================================================

\section{Continue on Better RSA Encryption Approach}
Apply $RSA_{N,e}$ on a random $x \leftarrow \mathbf{Z}_N^*$. Then, we know $x$ is hard to recover from $y=RSA_{N,e}(x)$.
\\
We first use a hash function on $x$ and encrypt message $m$:
\[c = (y=RSA_{N,e}(x) = x^e \bmod N, H(x) \oplus m)\]  
\\
$Dec(sk=(N,d), c=(y,p))$: Compute $x=RSA_{N,d}(y) = y ^d \bmod N$ and output $H(x) \oplus p$. This mechanism meets the correctness requirement.
\\
\textbf{Note}: Because $x$ is unknown, $H(x)$ would be close to completely unknown.
\vspace{8mm}
\\
We also need to check security requirement of RSA.
\\\\
\textbf{CPA Security}: Hash function (really random like)
\\
A good hash function "practically behaves" like a uniform random function (a.k.a random oracle) e.g.

\vspace{10mm}
%=============================================================================
%=============================================================================
\section{Digital Signature}

\vspace{10mm}
%=============================================================================
%=============================================================================
\section{Modeling Digital Signature}

\vspace{10mm}
%=============================================================================
%=============================================================================
\section{RSA Signature}

\vspace{10mm}
%=============================================================================
%=============================================================================

%\lipsum

%=============================================================================
%=============================================================================

\bibliographystyle{alpha}
% Uncomment below if you have any references:
%\bibliography{\jobname}

%=============================================================================
%=============================================================================

\end{document}
