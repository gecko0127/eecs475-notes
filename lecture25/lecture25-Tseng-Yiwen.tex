\documentclass{scribe}

%====================================================================
%====================================================================
% IMPORTANT: PLEASE UPDATE THE LECTURE INFORMATION BELOW:
\setcounter{lecture}{25} % Lecture number here, from The Big Table on Canvas
\renewcommand{\lectureTitle}{Digital signatures based on discrete log: Identification Schemes, Schnorr's identification, Fiat-Shamir}
\renewcommand{\lecturer}{Mahdi Cheraghchi}
\renewcommand{\scribe}{Yi-Wen Tseng}
\renewcommand{\lectureDate}{April 10, 2023} % Date of the lecture
%====================================================================
%====================================================================

\begin{document}

\maketitle

%=============================================================================
%=============================================================================
\section{Identification Schemes}
We can apply the hardness of discrete log to construct digital signatures,
 which is originally posted by Diffie Hellmen.
\\
It is tricky and requires a detour into so-called \textbf{Identification Scheme}.
\\
\subsection{Goal}
\textbf{Identification Scheme}: Prove that you have the key without revealing it.
\subsection{Set-Up}
We want to achieve that goal based on the hardness of discrete log.
\\
We have a huge group $G$ of known order $q$. $G$ is a cyclic group, with order $|G| = q$, and generator $g$:
\[\mathbb{G} = \{g^0,g^1,g^2, \dots, g^{q-1}\} = \]
Suppose $q$ is prime (for instance, think about $\mathbb{Z}_p^*$, $p =2q+1$ is prime.)
\\
We want:
\begin{itemize}
    \item Secrect key to be a random $x \in \mathbb{Z}_q$
    \item Public key would be $ y = g^x$
\end{itemize}
Schnorr provided an interactive protocol between a prover (who has $x$) and a verifier (who only has $y$)
\\
Prover wants to prove the knowledge of $x$ without revealing $x$.
%DRAW THE MODEL
We need to show the correctness and the soundness of this model:
\begin{itemize}
    \item \textbf{Correctness}:\\
            If $P,V$ run the protocol honestly, ($y = g^x$), then $V$ accepts.
            \[g^s = g^{rx+k} = g^k \cdot (g^x)^r = c \cdot y^r\]
    \item \textbf{Soundness}:\\
            We need to make sure that if $V$ accepts (w.h.p), then $P$ knows $x$.\\
            Thought experiment: Consider two challenges $r_1 \neg r_2$ sent by $V$ to $P$, for which $P$ menages to make $V$ accept.\\
            Say $S_1$, $S_2$ are the responses by $P$.\\
            We know that if $g^{S_1} = c \cdot y^{r_1}$ and $g^{S_2} = c \cdot y^{r_2}$, then
            \[g^{S_1 - S_2} = y^{r_1-r_2} = g^{x(r_1-r_2)}\]
            \[ s_1-s_2 = x(r_1-r_2) \bmod q\]
            \[x = (s_1 - s_2)(r_1-r_2)^{-1} \bmod q\]
            so we extract $x$ from the program runs $P$ against $V$.
\end{itemize}


\vspace{10mm}
\section{Zero Knowledge}
\textbf{Claim}: There is an efficient "simulator" that can efficiently sample from the distribution of the exchanged information without knowing $x$.
\\
\textbf{Trick}: Sample from the joint distribution ($c,r,s$) in this order: first $r$, then $s$, then $c$ in the way below:
\begin{enumerate}
    \item Choose $r \leftarrow \mathbb{Z}_q$ uniformly
    \item Choose $s \leftarrow \mathbb{Z}_q$ uniformly
    \item Set $c = g^s \cdot y^{-r} \in G$
\end{enumerate}
This has exactly the correct distribution but knows nothing about $x$ that $V$ doesn't.
In other words, $V$ learns nothing new about $x$ other than the fact that $P$ knows it.
\vspace{5mm}
There are two main issues in this construction:
\begin{enumerate}
    \item We want to sign stuff
    \item We do not want interaction
\end{enumerate}
\textbf{Fiat-Shamir Transform}: use a hash function
\\
Signature Scheme: $(Gen,Sign,Ver)$
\begin{itemize}
    \item Gen: Choose $x \leftarrow \mathbb{Z}_q$, output $(vk = y = g^x \in G, sk=x)$
    \item Sign($sk=x$, $m \in \{0,1\}^*$): Choose $k \leftarrow \mathbb{Z}_q$, let $c = g^k \in G$.
    \item Compute $r = H(m,c)$, where $H$ is really a "random oracle" and $s = k+r \cdot x \bmod q$. Output $\sigma = (r,s)$
    \item Ver($vk = y, m, \sigma = (r,s)$): compute $c = g^s \cdot y^{-r}$ and accept if $H(m,c) = r$.
\end{itemize}
\textbf{Theorem}: If discrete log is hard on $G$ and $H$ is a hash function modeled as a random oracle, then this is unforgeable.
\\
\textbf{Proof Idea}: Because $H$ is a random oracle, the values $r=H(m,c)$ that a forger must deal with like truly random challenges in Schnorr's ID protocol. A forger will not be able to answer a challenge unless:
\begin{enumerate}
    \item It gets extremely lucky in receiving the one challenge it knows how to handle
    \item It is able to compute $x=\log_{g}(y)$ for the legit signer's public key $y$. (a uniform random element of $G$).
\end{enumerate} 

%=============================================================================
%=============================================================================
%=============================================================================
%=============================================================================

%\lipsum

%=============================================================================
%=============================================================================

\bibliographystyle{alpha}
% Uncomment below if you have any references:
%\bibliography{\jobname}

%=============================================================================
%=============================================================================

\end{document}
