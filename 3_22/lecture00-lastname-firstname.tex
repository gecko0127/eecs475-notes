\documentclass{scribe}

%====================================================================
%====================================================================
% IMPORTANT: PLEASE UPDATE THE LECTURE INFORMATION BELOW:
\setcounter{lecture}{20} % Lecture number here, from The Big Table on Canvas
\renewcommand{\lectureTitle}{Elementary number theory: Abelian groups, cyclic groups, Diffie-Hellman key exchange Intro}
\renewcommand{\lecturer}{Mahdi Cheraghchi}
\renewcommand{\scribe}{Yi-Wen Tseng}
\renewcommand{\lectureDate}{March 22, 2023} % Date of the lecture
%====================================================================
%====================================================================

\begin{document}

\maketitle

%=============================================================================
%=============================================================================

\section{Number Thoery}


\section{Group Thoery}

\textbf{Definition}: $(G,\circ)$ where $\circ$ is a binary operation such that $\circ:G \times G \rightarrow G$ (we denote $\circ(g,h)$ as $g \cdot h$) is a group if:

\begin{enumerate}
    \item \textbf{Identity}: $\exists e \in G$ such that $\forall g \in G$: $e \circ g = g \circ e = g$
    \item \textbf{Inverse}: $\forall g \in G$, $\exists g^{-1} (or -g)$ such that $g \circ g^{-1} = e$
    \item \textbf{Associativity}: $\forall g_1, g_2, g_3 \in G$: $(g_1 \circ g_2) \circ g_3 = g_1 \circ (g_2 \circ g_3)$
    \item \textbf{Commutativity (Abelian Group)}: $\forall g,h \in G$, $g \circ h = h \circ g$
\end{enumerate}

\noindent\textbf{Example}: $(\mathbb{Z}_n \text{, } + \text{ (mod N)})$ is an Abelian Group
\begin{enumerate}
    \item \textbf{Identity}: $ a + 0 \text{ (mod N)} = 0 + a \text{ (mod N)} = a \text{ mod N}$
    \item \textbf{Inverse}: $a+ (-a) \text{ (mod N)} = 0 \text{ (mod N)}$
    \item \textbf{Associativity}: $(a+b)+c \text{ (mod N)} = a+(b+c) \text{ (mod N)} $
    \item \textbf{Commutativity}: $a+b \text{ (mod N)} = b+a \text{ (mod N)}$
\end{enumerate}

\noindent\textbf{Example}: $(\mathbb{Z}_n^{*} \text{, } \cdot \text{ (mod N)})$ is an Abelian Group
\begin{enumerate}
    \item \textbf{Identity}: $ a \cdot 1 \text{ (mod N)} = 1 \cdot a \text{ (mod N)} = a \text{ mod N}$
    \item \textbf{Inverse}: $a \cdot (a^{-1}) \text{ (mod N)} = 1 \text{ (mod N)}$
    \item \textbf{Associativity}: $(a \cdot b)\cdot c \text{ (mod N)} = a \cdot (b \cdot c) \text{ (mod N)} $
    \item \textbf{Commutativity}: $a \cdot b \text{ (mod N)} = b \cdot a \text{ (mod N)}$
\end{enumerate}
\indent *Note: If $a,b \in \mathbb{Z}_n^{*}$, we never get $a \cdot b=0 \text{ (mod N)}$.
\vspace{5mm}
\\\noindent\textbf{Notation}: $|G|$: group order.






\lipsum{}

%=============================================================================
%=============================================================================

\bibliographystyle{alpha}
% Uncomment below if you have any references:
%\bibliography{\jobname}

%=============================================================================
%=============================================================================

\end{document}
