\documentclass{scribe}

%====================================================================
%====================================================================
% IMPORTANT: PLEASE UPDATE THE LECTURE INFORMATION BELOW:
\setcounter{lecture}{23} % Lecture number here, from The Big Table on Canvas
\renewcommand{\lectureTitle}{RSA function and RSA encryption}
\renewcommand{\lecturer}{Mahdi Cheraghchi}
\renewcommand{\scribe}{Yi-Wen Tseng}
\renewcommand{\lectureDate}{April 3, 2023} % Date of the lecture
%====================================================================
%====================================================================

\begin{document}

\maketitle

%=============================================================================
%=============================================================================

\section{RSA Function}
In general public key encryption relies on some computational hardness for secure:
\begin{itemize}
    \item \textbf{Diffie-Hellman}: rely on the hardness of discrete log (finding an \textbf{unknown exponent} $a$ givne $g^a$) in a group of \textbf{known} order
    \item \textbf{RSA}: rely on the hardness of factoring and finding roots (finding \textbf{unkonwn base} but known exponent) in a group of \textbf{unknown} order
\end{itemize}

\subsection{Mathematical Foundation of RSA Function}
Let $N = p \cdot q$ be the product of two huge distinct prime.
\\
Then, $\mathbf{Z}_N^* = \{a \in \mathbf{Z}_N = \{0, \dots N-1\}: gcd(a,N) =1\}$
\\
We start with $\mathbf{Z}_N$ and throw out all multiples of $p$ $(0,p,2p, \dots , (q-1)p)$ and of $q$ $(0,q, \dots, (p-1)q)$ which would double count 0. 
\\
This means $|\mathbf{Z}_N^*| = \varphi(N) = p \cdot q - q -p +1 = (p-1)\cdot (q-1)$.
\\
\textbf{Euler's Theoreom}: In any group $G$, $\forall  a \in G$, $a^{|G|} = 1 \in G$. 
Say $G = \mathbf{Z}_N^*$, $a^{\varphi(N)}=a^{(p-1)(q-1)} = 1 (\bmod N)$ 
(Aruthmetic  mod N is "mod $\varphi(N)$" in the exponent)
\\
Take some $e$ such that $gcd(e,\varphi(N)) = 1$ (example: $e$ is prime such that $e \nmid (p-1)$, $e \nmid (q-1)$ this works)
\\
By Euclidean algorithm, we can compute integers $A$ and $B$, which are B\'{e}zout coefficients, such that $Ae+B\varphi(N) = 1$.
\\
\[\Rightarrow A \cdot e = 1 - B \varphi(N) = 1 (\bmod \varphi(N))\]
Define $d = A \bmod \varphi(N)$ is the multiplicative inverse of $e \bmod \varphi(N)$: $d=e^{-1} (\bmod \varphi(N))$ and $d \cdot e = 1 (\bmod \varphi(N))$. (usually choose $e=3$)
\vspace{10mm}

\subsection{RSA Function}
The choice of $N, e, d$ gives us the RSA function and its inverse.
\\
\textbf{Definition}: For $N=p \cdot q$ (large distinct primes p, q) and $e \in Z_{\varphi(N)}^*$ with $d = e^{-1} (\bmod \varphi(N))$, the RSA function $RSA _{N,e}: \mathbf{Z}_N^* \rightarrow \mathbf{Z}_N^*$ is a bijection and $RSA _{N,e}(x) = x^e \bmod N $.
\\
The inverse is $RSA _{N,d}(y) = RSA _{N,e}^{-1}(y) = y^d \bmod N  = x ^{ed (\bmod \varphi(N))} \bmod N = x ^1 = x \bmod N$

\section{RSA Encryption}

\lipsum

%=============================================================================
%=============================================================================

\bibliographystyle{alpha}
% Uncomment below if you have any references:
%\bibliography{\jobname}

%=============================================================================
%=============================================================================

\end{document}
