\documentclass{scribe}

%====================================================================
%====================================================================
% IMPORTANT: PLEASE UPDATE THE LECTURE INFORMATION BELOW:
\setcounter{lecture}{20} % Lecture number here, from The Big Table on Canvas
\renewcommand{\lectureTitle}{Elementary number theory: Abelian groups, cyclic groups, Diffie-Hellman key exchange Intro}
\renewcommand{\lecturer}{Mahdi Cheraghchi}
\renewcommand{\scribe}{Yi-Wen Tseng}
\renewcommand{\lectureDate}{March 22, 2023} % Date of the lecture
%====================================================================
%====================================================================

\begin{document}

\maketitle

%=============================================================================
%=============================================================================

\section{Number Thoery}
\subsection{Division}
In last class, we learned that:
\[a=b \text{ (mode N)} \Longleftrightarrow N|a-b \]
Assume $a=a' \text{ (mod N)}$ and $b=b' \text{ (mod N)}$, then
\begin{itemize}
    \item $a+b = a'+b' \text{ (mod N)}$
    \item $a-b = a'-b' \text{ (mod N)}$
    \item $a \cdot b = a' \cdot b' \text{ (mod N)}$
\end{itemize}
However, division is not always possible. For instace,
\[3 \cdot 2 = 15 \cdot 2 \text{ (mod 24)}\]
\[\text{But} 3\neq 15 \text{ (mod 24)}\]
\subsection{GCD}
\textbf{Invertability}: Given $b$, it's invertible if $\exists c$ such that $b \cdot c = 1 \text{ (mod N)}$, where $b$ and $c$ are multiplicative inverse.
\\\textbf{Lemma}: $b \le 1$, $N>1$, $b$ is invertible mod N if and only if $gcd(b,N)=1$
\\\textbf{Proof}: Assume $b \cdot c = 1 \text{ (mod N)}$
\[b \cdot c -1 = N \cdot q\]
\[bc-Nq=1\]
\[gcd(b,N) = 1\] becasue we learned that gcd is the smallest positive integer expressible in this way.
Also, if $gcd(b,N)=1$, then $\exists X,Y$ such that $X \cdot b + Y \cdot N = 1 \text{ (mod N)}$ and $X \cdot b =1 \text{ (mod N)}$
\\\textbf{Corollary}: Using Extended Euclidean, we can compute inverse mod N fast.
\vspace{5mm}
\\\textbf{Uniqueness}: If $c$, $c'$ are both inverse of b, then $c=c' \text{ (mod N)}$
\\Assume $b \cdot c = 1$, then we can write it as $N \text{ } |\text{ } b\cdot c-1$. Assume $b \cdot c' = 1$, then we can write it as $N \text{ } |\text{ } b\cdot c-'1$.
\\Then, $N \text{ } | \text{ } b \cdot (c-c')$. Because $gcd(N,b) = 1$, inverse exists. Thus,
\[N \text{ } | \text{ } (c-c')\] 
\[c=c' \text{ (mod N)}\]
\\Example: $b=11$, $N=17$
\[(-3) \cdot 11 + 2 \cdot 17 = 1\]
\[-3 \text{ (mod 17)} = 14 \text{ (mod 17)}\]
\[11 \cdot 14 = 1 \text{ (mod 17)}\]


\section{Group Thoery}

\subsection{Abelian Group}
\textbf{Definition}: $(G,\circ)$ where $\circ$ is a binary operation such that $\circ:G \times G \rightarrow G$ (we denote $\circ(g,h)$ as $g \cdot h$) is a group if:

\begin{enumerate}
    \item \textbf{Identity}: $\exists e \in G$ such that $\forall g \in G$: $e \circ g = g \circ e = g$
    \item \textbf{Inverse}: $\forall g \in G$, $\exists g^{-1} (or -g)$ such that $g \circ g^{-1} = e$
    \item \textbf{Associativity}: $\forall g_1, g_2, g_3 \in G$: $(g_1 \circ g_2) \circ g_3 = g_1 \circ (g_2 \circ g_3)$
    \item \textbf{Commutativity (Abelian Group)}: $\forall g,h \in G$, $g \circ h = h \circ g$
\end{enumerate}

\noindent\textbf{Example}: $(\mathbb{Z}_n \text{, } + \text{ (mod N)})$ is an Abelian Group
\begin{enumerate}
    \item \textbf{Identity}: $ a + 0 \text{ (mod N)} = 0 + a \text{ (mod N)} = a \text{ mod N}$
    \item \textbf{Inverse}: $a+ (-a) \text{ (mod N)} = 0 \text{ (mod N)}$
    \item \textbf{Associativity}: $(a+b)+c \text{ (mod N)} = a+(b+c) \text{ (mod N)} $
    \item \textbf{Commutativity}: $a+b \text{ (mod N)} = b+a \text{ (mod N)}$
\end{enumerate}

\noindent\textbf{Example}: $(\mathbb{Z}_n^{*} \text{, } \cdot \text{ (mod N)})$ is an Abelian Group
\begin{enumerate}
    \item \textbf{Identity}: $ a \cdot 1 \text{ (mod N)} = 1 \cdot a \text{ (mod N)} = a \text{ mod N}$
    \item \textbf{Inverse}: $a \cdot (a^{-1}) \text{ (mod N)} = 1 \text{ (mod N)}$  $\rightarrow$  a and $a^{-1}$ are coprime.
    \item \textbf{Associativity}: $(a \cdot b)\cdot c \text{ (mod N)} = a \cdot (b \cdot c) \text{ (mod N)} $
    \item \textbf{Commutativity}: $a \cdot b \text{ (mod N)} = b \cdot a \text{ (mod N)}$
\end{enumerate}
\indent *Note: If $a,b \in \mathbb{Z}_n^{*}$, we never get $a \cdot b=0 \text{ (mod N)}$. We have numbers in the group as always.
\vspace{5mm}
\\\noindent\textbf{Notation}: $|G|$: group order. 
\begin{itemize}
    \item  $(\mathbb{Z}_n \text{, } + )$ has order N
    \item  $(\mathbb{Z}_p^{*} \text{, } \cdot )$ (where p stands for prime) has order p-1 (from 1 to $p-1$).
\end{itemize}
\vspace{8mm}

\noindent\textbf{Theorem}: $G$ is a group and $m=|G|$. $\forall g \in G$, $g^m = \underbrace{(((g \circ g) \circ g) \circ g)}_{\text{m times}}=1$
\\\noindent\textbf{Proof}: For simplicity assume $G$ is abelian. Suppose $G = \{g_1, g_2, g_3, g_4, g_5, \dots\}$ and let $g \in G$ arbitrary.
Because $g \cdot g_i = g \cdot g_j \Rightarrow g_i = g_j$ (multiply by $g^{-1}$), the set $\{g \cdot g_i : i \in \{1, \dots, m\}\}$ covers all elements of $G$ exactly once.
\[g_1 \cdot g_2 \cdot g_3 \cdots g_m = (g \cdot g_1) \cdot (g \cdot g_2) \cdot (g \cdot g_3) \cdot \cdots \cdot (g \cdot g_m)\]  
\[g_1 \cdot g_2 \cdot g_3 \cdots g_m = g^{m} \cdot (g_1 \cdot g_2 \cdot g_3 \cdots g_m)\]
\[ 1 = g^{m}\]
\vspace{5mm}

\noindent\textbf{Corollary}: Fermat's Little Theorem: $\forall \text{ prime } p \text{, } gcd(a,p)=1 \text{ and } a^{p-1} = 1 \text{ (mod p)}$
\vspace{3mm}
\\\noindent\textbf{More General Theorem}: Euler's Theorem:
\[\varphi(N) = |\{a \text{ such that } 1 \le a \le N \text{,  } gcd(a,N) =1\}| \text{,  } |\mathbb{Z}_n^{*}| = \varphi(N)\]
\[\text{If } gcd(g,N) = 1 \text{  }\Rightarrow\text{  } g^{\varphi(N)} = 1 \text{ (mod N)}\]
\vspace{3mm}
\\\noindent\textbf{Corollary}: $m=|G|$, $\forall g \in G$, $\forall x \in \mathbb{Z}$. Because $g^m = 1$, $g^x = g^{x \text{ mod m}}$.
\vspace{3mm}
\\\noindent\textbf{Corollary}: $m = |G| > 1$, $e \in \mathbb{Z}$, $gcd(e,m) =1$. Define $d=e^{-1} \text{ (mod m)}$. Define function $f_e:G \rightarrow G$ as $f_e(g) = g^e$. Then, $f_e$ is a bijection whose inverse is $f_d$.
\\\\\noindent\textbf{Proof}:
\[f_d(f_e(g)) = f_d(g^e) = (g^e)^d = g^{e \cdot d} = g ^{e \cdot d \text{ (mod m)}} = g^{1} = g\]
\vspace{8mm}
\subsection{Cyclic Group}
\textbf{Definition}: $G$ is cyclic if $\exists g \in G$ such that
\[\{g^0=1, g^1, g^2,g^3, \dots, g^{m-1}\} = G\] (we say g generates G)
% ADD CYCLE DIAGRAM
\vspace{3mm}
\\\noindent\textbf{Non-example}
\[\mathbb{Z}_8^{*}=\{1,3,5,7\}\]
\[\text{power of } 1=\{1\}\]
\[\text{power of } 3=\{1, 3, 3^2 = 1, 3, \dots\}\]
\[\text{power of } 5=\{1, 5, 5^2 = 1, 5, \dots \}\]
\[\text{power of } 7=\{1, 7, 49=1, 7, \dots\}\]
\\\noindent\textbf{Example}: $\mathbb{Z}_p^{*}$ for prime $p$
\[\mathbb{Z}_7^{*}=\{1,2,3,4,5,6\}\]
\[\text{power of } 3=\{1, 3, 3^2 = 2, 3^3 = 6, 3^4 = 4, 3^5 = 5, 3^6 = 1\} \Rightarrow \text{ 3 generates }\mathbb{Z}_7^{*}\]
\[\text{power of } 2=\{1, 2, 2^2 = 4, 2^3 = 1, \dots \} \Rightarrow \text{ 2 does not generates }\mathbb{Z}_7^{*}\]
$G^{'} \subseteq G$ is a subgroup if $(G^{'}, \cdot )$ is a group. When g does not generate $G$, it generates a subgroup.
\textbf{Lagrange's Theorem}: If $G^{'} \subseteq G$ is a subgroup then,
\[|G^{'}| \text{ } \bigg| \text{ } |G|\]
\vspace{3mm}
\textbf{Fast Exponentiation}: Suppose we have an element g. Want to compute $g^M$.
\\Naive method: $g^{M} = \underbrace{g \cdot g \cdot g \cdots g}_{\text{M times}} = \underbrace{(((g \cdot g) \cdot g) \cdot g) \cdots}_{\text{M times}} $ %ADD EXPLANATION ABOUT WHY NOT TAKE THIS
\\Observe: $g^{2^{m}} = g^{2^{m-1}} \cdot g^{2^{m-1}} = (g^{2^{m-1}})^2$ 
\\\\If $M=2^m$, 
\[ g, g^2, g^4, g^8, g^{16}, \dots \]
How many operations we perform in this case? $T(M)$
\[T(M) = T(\frac{M}{2}) +1 \Rightarrow T(M) = \log{M} = M\] which is efficient
\\In general, $M = \sum_{i=0}^{l}m_i \cdot 2^{i}$. We get $g^{M} = \prod_{i=0}^{l} g^{2^{i}}$ by applying the trick above for each $g^{2^{i}}$. If $M$ is $l$ bits long, there are $O(l^2)$ multiplications altogether.
\vspace{5mm}
\\\noindent\textbf{Corollary}: Fast exponentiation allows us to compute inverses very fast because $g^{-1} = g^{|G|-1}$ (since $g^{|G|} = 1$). However, for $\mathbb{Z}_p^{*}$, we have a faster method: Extended Euclidean.
\\We now know how to compute $g^m$ from $g$ efficiently. Do we know how to compute $m$ from $g^m$? Discrete log: $m="log" g^m$ is conjectured to be extremely difficult. We use the difficulty in calculating this discrete log on constructing the \textbf{Diffie Hellman key exchange} mechanism.









%\lipsum{}

%=============================================================================
%=============================================================================

\bibliographystyle{alpha}
% Uncomment below if you have any references:
%\bibliography{\jobname}

%=============================================================================
%=============================================================================

\end{document}
