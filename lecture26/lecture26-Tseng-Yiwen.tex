\documentclass{scribe}

%====================================================================
%====================================================================
% IMPORTANT: PLEASE UPDATE THE LECTURE INFORMATION BELOW:
\setcounter{lecture}{26} % Lecture number here, from The Big Table on Canvas
\renewcommand{\lectureTitle}{Post-quantum and lattice-based cryptography}
\renewcommand{\lecturer}{Mahdi Cheraghchi}
\renewcommand{\scribe}{Yi-Wen Tseng}
\renewcommand{\lectureDate}{April 12, 2023} % Date of the lecture
%====================================================================
%====================================================================

\begin{document}

\maketitle

%=============================================================================
%=============================================================================
\section{Introduction}
In 1994, Peter Shor proposed a quantum algorithm that can factorize integers in polynomial time.
The idea behind this algorithm is to use the complex probabilities (quantum weidness).
The development of algorithm means that we can use it to break RSA which relies on the hardness of factorizing.
\\\\
Also, quantum search is weid. Based on Grover's algorithm, we can search in an unstructured array of size $N$ using $O(N)$ operations.
In crypto, quantum computers can brute force for a key of length $N$ in time $2^{\frac{N}{2}}$ rather than $2^N$. The remedy is to double the key size, so the computational complexity is $O(2^{\frac{2N}{2}}) =O(2^N)$.
\\\\
Shor's algorithm also computes Discrete log in polynomial time. This means that it can break Diffie-Hellmen, El-Gamal, etc.
\\\\
Thus, post-quantum cryptography can not rely on the hardness of factoring and discrete log. The older proposal is to rely on the hardness of solving subset sum or hash functions. The more successful proposals rely on \textbf{Coding Theory} and \textbf{Lattice Theory} instead.
\vspace{10mm}
%=============================================================================
%=============================================================================
\section{Lattice Based Crypto}
\textbf{Idea}: Build crypto based on hardness of problems about lattices
\\\\
\textbf{Lattice}: a periodic, infinite grid in $n$-dim space $\mathbb{R}^n$. It is generated by $n$ basis vectos in $\mathbb{R}^n$. It is same as vector space, but with integer linear combinations of vectors only.
% ADD GRAPH
\[ \mathcal{B} = \{b_1,b_2, \dots, b_n\}\]
\[ \mathcal{L\{\mathcal{B}\}} = \{z_1 \cdot b_1+z_2 \cdot b_2+z_n \cdot b_n:z_i \in \mathbb{Z}\}\]
in matrix form: $\mathcal{L\{\mathcal{B}\}} = \{B\cdot z: z\in \mathbb{Z}^n\} \subseteq \mathbb{R}^n$
% TYPE OUT MATRIX FORM
\\\\
Below is a list of conjectured lattice problems:
\begin{enumerate}
    \item Shortest Vector Problem (SVP): Given $B$, find the shortest (or a "very short") nonzero vectr $v \in \mathbb{L}(B)$
    \item Decoding (a.k.a Closest Vector or CVP): Given $B$, and a target point $t \in \mathbb{R}^n$, find teh closest vector in $\mathbb{L}(B)$ to $t$.
    \item Learning With Errors (LWE):\\
            \begin{itemize}
                \item Fix $n$ (dimension)
                \item Fix $q \approx n^2$
            \end{itemize}
            Pick a secret $s \leftarrow \mathbb{Z}_q^n$, an instance of LWE is to find the secret $s$ in the construction below given $n,q,A,b$:
            % ADD GRAPH or MATRIX
            which is conjectured to be hard.
\end{enumerate}
\section{Learning With Errors}
\subsection{LWE and CVP}
LWE is closely related to CVP: $b \approx A \cdot s$ is a target vector close to lattice point $v = A \cdot s$, where $A$ is the lattice basis.
\\\\
\textbf{Decision LWE}: Given $(A,b)$, distinguish between
\begin{enumerate}
    \item ($A$,$b \approx A \cdot s$)
    \item ($A$,$b$) uniform random
\end{enumerate}
\vspace{5mm}
\textbf{Theorem}: LWE search and decision are equivalent under ppt reduction. (each can be solved efficiently iff the other can)
\vspace{10mm}
%=============================================================================
%=============================================================================
\subsection{LWE for Key Exchange}
%ADD GRAPH
Alice and Bob can agree on a bit in the end.
\\
This can be turned into p.k.e just like turing Diffie-Hellmen to El-Gamal.
\\
To encrypt a bit $m \in \{0,1\}$, Alice can compute $c \approx r^t \cdot u + m \cdot (\frac{q}{2})$
\\
To recover m: compute $p = c - v^t \cdot s \approx \underbrace{r^t \cdot A \cdot s}_{k_A}+m(\frac{q}{2}) - \underbrace{r^t \cdot A \cdot s}_{k_B} = m(\frac{q}{2})$
can recover $m$ by seeing if $p$ is closer to $0$ or $\frac{q}{2}$.
\\\\
\textbf{Theorem}: This key agreement/PKE is CPA secure assuming LWE decision is hard.

%=============================================================================
%=============================================================================


%\lipsum

%=============================================================================
%=============================================================================

\bibliographystyle{alpha}
% Uncomment below if you have any references:
%\bibliography{\jobname}

%=============================================================================
%=============================================================================

\end{document}
